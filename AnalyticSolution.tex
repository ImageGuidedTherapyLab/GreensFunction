%
%  This simple example illustrates how documents can be
%  split into smaller segments, each segment processed
%  by latex2html separately.  This document can be
%  processed through latex and latex2html with the
%  corresponding makefile.
%

\documentclass{article}         % Must use LaTeX 2e
\usepackage[plainpages=false, colorlinks=true, citecolor=black, filecolor=black, linkcolor=black, urlcolor=black]{hyperref}		
\usepackage[left=.75in,right=.75in,top=.75in,bottom=.75in]{geometry}
\usepackage{makeidx,color,boxedminipage}
\usepackage{graphicx,float}
\usepackage{amsmath,amsthm,amsfonts,amscd,amssymb} 

%%%%%%%%%%%%%%%%%%%%%%%%%%%%%%%%%%%%%%%%%%%%%%%%%%%%%%%%%%%%%%%%%%%%%%
%	Some math support.					     %
%%%%%%%%%%%%%%%%%%%%%%%%%%%%%%%%%%%%%%%%%%%%%%%%%%%%%%%%%%%%%%%%%%%%%%
%
%	Theorem environments (these need the amsthm package)
%
%% \theoremstyle{plain} %% This is the default

\newtheorem{thm}{Theorem}[section]
\newtheorem{cor}[thm]{Corollary}
\newtheorem{lem}[thm]{Lemma}
\newtheorem{prop}[thm]{Proposition}
\newtheorem{ax}{Axiom}

\theoremstyle{definition}
\newtheorem{defn}{Definition}[section]

\theoremstyle{remark}
\newtheorem{rem}{Remark}[section]
\newtheorem*{notation}{Notation}
\newtheorem*{exrcs}{Exercise}
\newtheorem*{exmple}{Example}

%\numberwithin{equation}{section}


%%%%%%%%%%%%%%%%%%%%%%%%%%%%%%%%%%%%%%%%%%%%%%%%%%%%%%%%%%%%%%%%%%%%%%
%	Macros.							     %
%%%%%%%%%%%%%%%%%%%%%%%%%%%%%%%%%%%%%%%%%%%%%%%%%%%%%%%%%%%%%%%%%%%%%%
%
%	Here some macros that are needed in this document:

\newcommand{\motion}{\mathbf{\varphi}}
\newcommand{\hmotion}{\mbox{\boldmath $\hat{\varphi}$}}
\newcommand{\cauchy}{\mbox{\boldmath $\sigma$}}
\newcommand{\eqn}[1]{(\ref{#1})}
\newcommand{\hOmega}{\hat{\Omega}}
\newcommand{\homega}{\hat{\omega}}
\newcommand{\nphalf}{n+\frac{1}{2}}
\newcommand{\nmhalf}{n-\frac{1}{2}}
\newcommand{\kmhalf}{k-\frac{1}{2}}
\newcommand{\kphalf}{k+\frac{1}{2}}
\newcommand{\picdir}{pdffig}

\begin{document}                % The start of the document

The governing equations for LITT are derived from Pennes bioheat
equation coupled with a source term provided by laser tissue
interaction. 
\begin{equation} \label{BioheatPDE}
  \partial u_t   
 -\nabla \cdot  (   \alpha \nabla u  ) 
 + \frac{\omega c_\textit{blood}}{\rho c_p} (u  - u_a )
 = \frac{1}{\rho c_p} \; q_{laser}
     \qquad \text{in } U \backslash U_{tip}
\end{equation}
\[
   q_{laser}(x,t)  = 
    \int_{U_{tip}}
 \frac{p(t) \mu_\textit{eff}^2}{\text{Vol}(U_{tip}) }  
   \frac{ \exp(-\mu_\textit{eff} \| x -{ \xi}\|) }
      {4\pi \| x-{ \xi}\|} \; d\xi
   \quad   x \in U \backslash U_{tip}
\]
\[
  \alpha  = \frac{k}{\rho c_p}
\quad
  \mu_{eff}  = \sqrt{ 3 {\mu_a} \mu_{tr} }
\quad
 \mu_{tr} = \mu_a  + \mu_s (1-g)
\]
\[
n \cdot ( \alpha \nabla u  )  = 0 
                       \qquad \text{on } \partial U
\qquad
n \cdot ( \alpha \nabla u  )  = h (u - u_\textit{cooling}) 
                       \qquad \text{on } \partial U_\textit{tip}
\]

Previous sensitivity analysis have indicated that uncertainty in 
the optical parameters, $ \omega < k < \mu_\text{eff}$, 
provide the largest affect on the final temperature prediction outcomes.

%%%%%%%%%%%%%%%%%%%%%%%%%%%%%%%%%%%%%%%%%%%%%%%%%%%%%%%%%%%%%%%%%%%%%%%%
\section{Analytic Steady State Solution}\label{AnalyticBioheatModel}
%%%%%%%%%%%%%%%%%%%%%%%%%%%%%%%%%%%%%%%%%%%%%%%%%%%%%%%%%%%%%%%%%%%%%%%%

A 1D spherically symmetric radial decomposition 
of the solution,
$u(r)~-~u_a~=~\frac{\mathcal{R}(r)}{r}$, simplifies the analysis of the differential operator in
spherical coordinates.
\[
    \frac{\omega c_\textit{blood}}{\rho c_p}  \frac{ \mathcal{R}(r)}{r} 
   -
    \frac{\alpha }{r}   \frac{d^2 \mathcal{R}}{dr^2}
   = 
    \frac{\mu_\textit{eff}^2 \;  P}{ \; \rho c_p}   
    \frac{ \exp \left(-\mu_\textit{eff} \; r \right) }{ 4 \; \pi \; r}
\]
\[
   u(r_1) = u_0
  \qquad
  \left. \frac{d u}{dr} \right|_{r_2}  = 0
  \qquad
   r_1 < r_2
\]
Constant coeffients are assumed. A 1D spherically symmetric radial decomposition 
of the solution
simplifies the analysis of the differential operator in spherical coordinates.
A particular solution, $u_p$, will be used to construct a solution. 
\[
 -\Delta  \frac{u_p(r)}{r} 
 + \frac{\omega c_\text{blood}}{k}  \frac{u_p(r)}{r} 
 = \frac{1}{k} \; q_{laser}
\qquad
\qquad
  \frac{u_p(r)}{r}  = 
    \frac{\mu_\text{eff}^2 \; P \; \exp \left(-\mu_\text{eff} \; r \right) }
         { 4 \; \pi \; r \; \left(\omega c_\text{blood}-k \; \mu_\text{eff}^2 \right)} 
\]
Notice that in spherical coordinates
\[
  \frac{d}{dr}  \frac{u_p(r)}{r} 
   = 
    \frac{u_p'(r)}{r}  - \frac{u_p(r)}{r^2}   
  \qquad
  \qquad
 \Delta  \frac{u_p(r)}{r} 
   = 
  \frac{1}{r^2} 
  \frac{d}{dr} \left( r^2 \frac{u_p(r)}{r}  \right)
   = 
    \frac{u_p''(r)}{r} 
\]
%FIXME { \color{red} May also consider Robin BC 
%\[
%  \left. k  \frac{d u}{dr} \right|_{r_1}  = h (u - u_\infty) 
%\]
%} 
From classical theory~\cite{boyce1992elementary}, 
the general solution is the linear combination of the homogeneous solution, $u_h$, and
a particular solution, 
$ u_p = \frac{1}{r} \left( A \exp (-\mu_\textit{eff} \; r) + B \; r \;  \exp
(-\mu_\textit{eff} \; r) \right) $.
In this case, the particular solution was obtained from the method of undetermined
coefficients for $A,B \in \mathbb{R}$. 
\begin{equation}\label{OneDSolution}
u = 
  \underbrace{
    C_1 \frac{\exp\left( \sqrt{\frac{\omega c_\textit{blood}}{k}} \; r \right)}{r}  
  + 
    C_2 \frac{\exp\left(-\sqrt{\frac{\omega c_\textit{blood}}{k}} \; r \right)}{r}  
    }_{u_h}
  + 
  \underbrace{
    \frac{\mu_\textit{eff}^2 \; P \; \exp \left(-\mu_\textit{eff} \; r \right) }
         { 4 \; \pi \; r \; \left(\omega c_\textit{blood}-k \; \mu_\textit{eff}^2 \right)} 
    }_{u_p}
  + 
    u_a
\end{equation}
The boundary conditions are used to determine the coefficients of the homogeneous solution.
Applicator cooling is specified by the boundary condition at $r = r_1$.
The domain is assumed large enough that no heat flux is observed at the far boundary $r = r_2$.
\[
\begin{bmatrix}
    \frac{\exp\left( \sqrt{\frac{\omega c_\textit{blood}}{k}} \; r_1 \right)}{r_1}  
  & 
    \frac{\exp\left(-\sqrt{\frac{\omega c_\textit{blood}}{k}} \; r_1 \right)}{r_1}  
  \\
  \left.
   \frac{d}{dr}
    \frac{\exp\left( \sqrt{\frac{\omega c_\textit{blood}}{k}} \; r \right)}{r}  
  \right|_{r_2}
  & 
  \left.
   \frac{d}{dr}
    \frac{\exp\left(-\sqrt{\frac{\omega c_\textit{blood}}{k}} \; r \right)}{r}  
  \right|_{r_2}
\end{bmatrix}
\begin{bmatrix}
  C_1 \\
  C_2 \\
\end{bmatrix}
= 
\begin{bmatrix}
      u_0 - u_p(r_1) - u_a    \\
  - \left. \frac{d u_p}{dr} 
  \right|_{r_2}\\
\end{bmatrix}
\]
%The 1D solution provides an estimate of the heating from a single point source
%with applicator boundary at $r_1$. The Symbolic Toolbox of MATLAB was used to determine
%and verify all coefficients and \verb#ccode# was used to write out the kernel.
%Similar analytical solutions are provided in \cite{Giordano2010,Vyas1992,Deng2002}.
The 1D solution provides an estimate of the heating from a single point source
with applicator boundary at $r_1$. Mathematica 7 (Wolfram, Champaign, IL) was used
to determine and verify all coefficients. Then, \verb#ccode# was used to write out the kernel.
Similar analytical solutions are provided in \cite{Giordano2010,Vyas1992,Deng2002}.
%FIXME \text{ \color{red} (Do we get the same result ?) } 
Heating caused by the cylindrical geometry of the diffusing tip was modeled as
evenly distributed point sources, $M = 10$, along the axial dimension of the
applicator at positions $r_{0_i}$. 
\begin{equation}\label{GreenSuperPosition}
  u = \sum_{i=1}^M u_h( r-r_{0_i}) + u_p( r-r_{0_i})
\end{equation}

%%%%%%%%%%%%%%%%%%%%%%%%%%%%%%%%%%%%%%%%%%%%%%%%%%%%%%%%%%%%%%%%%%%%%%%%
\subsection{Steady State Sensitivity}
%%%%%%%%%%%%%%%%%%%%%%%%%%%%%%%%%%%%%%%%%%%%%%%%%%%%%%%%%%%%%%%%%%%%%%%%

The sensitivity of the solution to the optical properties, $\frac{\partial
u}{\partial \mu_\text{eff}}$ is also of
interest.
Sensitivities are obtain through chain rule differentiation of the governing
PDE.
\[
\frac{\partial}{\partial \mu_\text{eff}}
\left[
 -\nabla \cdot \left(   \alpha \nabla u  \right) 
 + \frac{\omega c_\textit{blood}}{\rho c_p} (u  - u_a )
 = \frac{1}{\rho c_p} \; q_{laser}
\right]
\qquad
\Rightarrow
\qquad
 -\nabla \cdot  \left(   \alpha \nabla  \frac{\partial u}{\partial \mu_\text{eff}} \right) 
 + \frac{\omega c_\textit{blood}}{\rho c_p} \frac{\partial u}{\partial \mu_\text{eff}}  
 = \frac{1}{\rho c_p} \;
   \frac{\partial q_{laser}}{\partial \mu_\text{eff}}
\]
Assuming constant coefficients, 
a 1D spherically symmetric radial decomposition 
of the solution,
$ \frac{\partial u}{\partial \mu_\text{eff}} = \frac{\mathcal{R}(r)}{r}$, again simplifies the analysis of the differential operator in
spherical coordinates.
\[
   L \left[\frac{\partial u}{\partial \mu_\text{eff}}\right] = 
    \omega c_\textit{blood}  \frac{ \mathcal{R}(r)}{r} 
   -
    \frac{k }{r}   \frac{d^2 \mathcal{R}}{dr^2}
   = 
   \frac{\partial q_{laser}}{\partial \mu_\text{eff}}
\]
\[
 \left.
 \frac{\partial u}{\partial \mu_\text{eff}} 
  \right|_{r_1}
  = 
 \frac{\partial u_0}{\partial \mu_\text{eff}} = 0
  \qquad
  \left. \frac{\partial }{\partial \mu_\text{eff}}
   \frac{d u}{dr} \right|_{r_2}  = 0
  \qquad
   r_1 < r_2
\]
From classical theory~\cite{boyce1992elementary}, 
the general solution is the linear combination of the homogeneous solution and
a particular solution. 
\[
 \frac{\partial u}{\partial \mu_\text{eff}} = 
  \frac{1}{r}
   \left(
  u_h(r)+ 
  u_p(r)
%%  +\underbrace{
%% \frac{\mu_\textit{eff} \;  P }{ 2 \; \pi \; \rho c_p}  
%% \left( u_{p_1} - u_{p_2} \right) 
%%  }_{\equiv u_p}
 \right)
\]
We will construct a solution using the linearity of the operator, $L$.
\[
  L \left[\frac{u_h}{r} \right] =  0 
  \qquad
  L \left[\frac{u_p}{r} \right] =  
   \frac{\partial q_{laser}}{\partial \mu_\text{eff}}
 %% L \left[\frac{u_{p_1}}{r} \right] = 
 %%   \frac{ \exp \left(-\mu_\textit{eff} \; r \right) }{  r}
 %% \qquad
 %% L \left[\frac{u_{p_2}}{r} \right] = 
 %%   \frac{ \mu_\textit{eff} \exp \left(-\mu_\textit{eff} \; r \right) }{ 2 }
 %%  \qquad
\]
%% \[
%%   L \left[ \frac{\partial u}{\partial \mu_\text{eff}}\right]= 
%%     \frac{\mu_\textit{eff} \;  P }{ 2 \; \pi \; \rho c_p}   
%%     \left(
%%     \frac{ \exp \left(-\mu_\textit{eff} \; r \right) }{  r}
%%    - 
%%     \frac{ \mu_\textit{eff} \exp \left(-\mu_\textit{eff} \; r \right) }{ 2 }
%%     \right)
%% \]
%% \[
%%  -\Delta  \frac{u_p(r)}{r} 
%%  + \frac{\omega c_\text{blood}}{k}  \frac{u_p(r)}{r} 
%%  = \frac{1}{k} \; q_{laser}
%% \qquad
%% \qquad
%%   \frac{u_p(r)}{r}  = 
%%     \frac{\mu_\text{eff}^2 \; P \; \exp \left(-\mu_\text{eff} \; r \right) }
%%          { 4 \; \pi \; r \; \left(\omega c_\text{blood}-k \; \mu_\text{eff}^2 \right)} 
%% \]
%% %FIXME { \color{red} May also consider Robin BC 
%% %\[
%% %  \left. k  \frac{d u}{dr} \right|_{r_1}  = h (u - u_\infty) 
%% %\]
%% %} 
%% The particular solution $u_{p_1}$ may be obtain from the
%% method of undetermined coefficients.
%% \[
%%   L \left[\frac{u_{p_1}}{r} \right] = 
%%  -\Delta  \frac{u_{p_1}(r)}{r} 
%%  + \frac{\omega c_\text{blood}}{k}  \frac{u_{p_1}(r)}{r} 
%%  = 
%%     \frac{ \exp \left(-\mu_\textit{eff} \; r \right) }{  r}
%% \qquad
%% \qquad
%%   \frac{u_{p_1}(r)}{r}  = 
%%     \frac{\exp \left(-\mu_\text{eff} \; r \right) }
%%          { r \; \left(\omega c_\text{blood}-k \; \mu_\text{eff}^2 \right)} 
%% \qquad
%% \text{\color{red} double check}
%% \]
The particular solution $u_p$ may be obtain from 
\[
\begin{split}
  L \left[\frac{u_p}{r} \right] = 
 -\Delta  \frac{u_p(r)}{r} 
 + \frac{\omega c_\text{blood}}{k}  \frac{u_p(r)}{r} 
 = 
    \frac{\partial q_{laser}}{\partial \mu_\text{eff}}
   \frac{\partial q_{laser}}{\partial \mu_\text{eff}}
 &  = 
   \frac{\partial }{\partial \mu_\text{eff}}
   \left(
    \mu_\textit{eff}^2 \;  P
    \frac{ \exp \left(-\mu_\textit{eff} \; r \right) }{ 4 \; \pi \; r}
   \right)
\\
 & = 
   2 \; \mu_\textit{eff} \;  P
    \frac{ \exp \left(-\mu_\textit{eff} \; r \right) }{ 4 \; \pi \; r}
   + 
    \mu_\textit{eff}^2 \;  P
    \left(-r\right) \frac{ \exp \left(-\mu_\textit{eff} \; r \right) }{ 4 \; \pi \; r}
 \\
 & = 
    \mu_\textit{eff} \;  P
    \frac{ \exp \left(-\mu_\textit{eff} \; r \right) }{ 2 \; \pi \; r}
   - 
    \mu_\textit{eff}^2 \;  P
                    \frac{ \exp \left(-\mu_\textit{eff} \; r \right) }{ 4 \; \pi }
 \end{split}
\]
% \frac{\partial u}{\partial \mu_\text{eff}}  
\[
  \frac{u_p(r)}{r}  =  
\text{\color{red} ?}
\]
As before, the homogeneous coefficients, $C_1, C_2$ must be determined from the 
boundary conditions.
\[
  L \left[\frac{u_h}{r} \right] =   
 L\left[
    C_1 \frac{\exp\left( \sqrt{\frac{\omega c_\textit{blood}}{k}} \; r \right)}{r}  
  + 
    C_2 \frac{\exp\left(-\sqrt{\frac{\omega c_\textit{blood}}{k}} \; r \right)}{r}  
\right]
  = 0
\]
\[
\begin{bmatrix}
    \frac{\exp\left( \sqrt{\frac{\omega c_\textit{blood}}{k}} \; r_1 \right)}{r_1}  
  & 
    \frac{\exp\left(-\sqrt{\frac{\omega c_\textit{blood}}{k}} \; r_1 \right)}{r_1}  
  \\
  \left.
   \frac{d}{dr}
    \frac{\exp\left( \sqrt{\frac{\omega c_\textit{blood}}{k}} \; r \right)}{r}  
  \right|_{r_2}
  & 
  \left.
   \frac{d}{dr}
    \frac{\exp\left(-\sqrt{\frac{\omega c_\textit{blood}}{k}} \; r \right)}{r}  
  \right|_{r_2}
\end{bmatrix}
\begin{bmatrix}
  C_1 \\
  C_2 \\
\end{bmatrix}
= 
\begin{bmatrix}
       - u_p(r_1)     \\
  - \left. \frac{d u_p}{dr} 
  \right|_{r_2}\\
\end{bmatrix}
\]
%%%%%%%%%%%%%%%%%%%%%%%%%%%%%%%%%%%%%%%%%%%%%%%%%%%%%%%%%%%%%%%%%%%%%%%%
\section{Transient Analytic Solution}
%%%%%%%%%%%%%%%%%%%%%%%%%%%%%%%%%%%%%%%%%%%%%%%%%%%%%%%%%%%%%%%%%%%%%%%%

A simplified analytic solution of the Pennes bioheat equation 
may also be considered as a surrogate model for the therapy planning.
\begin{equation} \label{TransientAnalytic}
  \frac{1}{\alpha}
  \partial u_t   
 -\Delta u   
 + \frac{\omega c_\text{blood}}{k} (u  - u_a )
 = \frac{1}{k} \; q_{laser}
     \qquad \text{in } U \backslash U_{tip}
\end{equation}
\[
  \left. k  \frac{d u}{dr} \right|_{r_1}  = h (u - u_\infty) 
  \qquad
  \left. \frac{d u}{dr} \right|_{r_2}  = 0
  \qquad
   r_1 < r_2
  \qquad
  \qquad
   u(r,0) = u_0
\]
The analytical form of the separation of variables chosen anticipates the
approaching a steady state solution.
\begin{equation} \label{AssumedAnalyticForm}
\begin{split}
 u(r,t) - u_a = T(t) \frac{u_h(r)}{r} + \frac{u_p(r)}{r} 
\\
 \dot{u}(r,t)  = \dot{T}(t) \frac{u_h(r)}{r}
\\
 \Delta u(r,t)  = T(t) \frac{u_h''(r)}{r }+ \frac{u_p''(r)}{r} 
\end{split}
\end{equation}
Substituting \eqn{AssumedAnalyticForm} in \eqn{BioheatPDE}
\[
  \frac{1}{\alpha}
  \dot{T}(t) \frac{u_h(r)}{r}
 -
 T(t) \frac{u_h''(r)}{r }
 + \frac{\omega c_\text{blood}}{k} 
  T(t) \frac{u_h(r)}{r} 
 = 0
\]
Following the classical separation of variable approach
\[
  \frac{\dot{T}(t)}{\alpha T(t)}
   =
  \frac{u_h''(r)
       - \frac{\omega c_\text{blood}}{k} u_h(r) 
       }{u_h(r) }
   =
   - \lambda
\]
yeilds two ODE's.
\[
  \dot{T}(t)+ \lambda \alpha T(t)  = 0 
   \qquad \Rightarrow \qquad
   T(t)  = C_0 \exp\left( -\lambda \alpha t \right)
\]
\[
  u_h''(r) + ( \lambda - \frac{\omega c_\text{blood}}{k} )  u_h(r)  = 0 
   \qquad \Rightarrow \qquad
  u_h(r)   = C_1 \sin \sqrt{ \lambda - \frac{\omega c_\text{blood}}{k} } r
           + C_2 \cos \sqrt{ \lambda - \frac{\omega c_\text{blood}}{k} } r
\]

The initial conditions and boundary conditions are used to determine the coefficients of the homogenious solution.
Applicator cooling is specified by the boundary condition at $r = r_1$.
The domain is assumed large enough that no heat flux is observed at the far boundary $r = r_2$.
\[
 u(r,0) - u_a = T(0) \frac{u_h(r)}{r} + \frac{u_p(r)}{r}  = u_0
\]
\[
T(t)
\begin{bmatrix}
  \left.
   \frac{d}{dr}
    \frac{\sin \sqrt{ \lambda - \frac{\omega c_\text{blood}}{k} } r}{r}  
  \right|_{r_1}
    - \frac{h}{k} 
    \frac{\sin \sqrt{ \lambda - \frac{\omega c_\text{blood}}{k} } r}{r}  
  & 
  \left.
   \frac{d}{dr}
    \frac{\cos \sqrt{ \lambda - \frac{\omega c_\text{blood}}{k} } r}{r}  
  \right|_{r_1}
    - \frac{h}{k} 
    \frac{\cos \sqrt{ \lambda - \frac{\omega c_\text{blood}}{k} } r}{r}  
  \\
  \left.
   \frac{d}{dr}
    \frac{\sin \sqrt{ \lambda - \frac{\omega c_\text{blood}}{k} } r}{r}  
  \right|_{r_2}
  & 
  \left.
   \frac{d}{dr}
    \frac{\cos \sqrt{ \lambda - \frac{\omega c_\text{blood}}{k} } r}{r}  
  \right|_{r_2}
  \\
\end{bmatrix}
\begin{bmatrix}
  C_1 \\
  C_2 \\
\end{bmatrix}
= 
\begin{bmatrix}
  \frac{h}{k} (u_p(r_1) - u_a  - u_\infty)       
  - \left. \frac{d u_p}{dr} 
  \right|_{r_1}\\
  - \left. \frac{d u_p}{dr} 
  \right|_{r_2}\\
\end{bmatrix}
\]
{\color{red} The time function $T(t)$ is differenct than the classical
approach. In this case the coefficients $C_1(t)$ and $C_2(t)$ depend on
time.} 
Heating caused by the cylindrical geometry of the diffusing tip was modeled as
evenly distributed point sources, $N=10$, along the axial dimension of the
applicator at positions $r_{0_i}$. 
\begin{equation}\label{GreenSuperPosition}
  u = \sum_{i=1}^N u_h( r-r_{0_i}) + u_p( r-r_{0_i})
\end{equation}
%%%%%%%%%%%%%%%%%%%%%%%%%%%%%%%%%%%%%%%%%%%%%%%%%%%%%%%%%%%%%%%%%%%%%%%%%%%
\section{PC Expansion Selection }
%%%%%%%%%%%%%%%%%%%%%%%%%%%%%%%%%%%%%%%%%%%%%%%%%%%%%%%%%%%%%%%%%%%%%%%%%%%
A Karhunen Loeve decomposition of the uncertain parameters is assumed.
The parameters are written as a linear combination of random perturbations 
around a mean function. $(Z_1,Z_2,...Z_d)$ are assumed \textit{mutually
independent} random variables with
known probability distributions, ie $Z_k = \mathcal{N}(m_k,\sigma_k)$ or $Z_k = \mathcal{U}(a_k,b_k)$,
\[
\begin{split}
k = k_0(x) + \sum_{k=1}^d k_k(x)  Z_k
\\
\omega = \omega_0(x) + \sum_{k=1}^d \omega_k(x)  Z_k
\\
q = q_0(x) + \sum_{k=1}^d q_k(x)  Z_k
\end{split}
\]
Following polynomial chaos theory,
the solution to \eqn{BaseModelOverview} may be approximated in terms of
a basis of orthogonal polynomials truncated at order $N$
\[
u(x,t,Z)  \approx  \sum_k^M u_k(x,t) \Phi_k(Z)
\qquad
M = 
\left(
\begin{split}
N & + d
\\
  & d
\end{split}
\right)
\qquad
\mathbb{E} \left[ \Phi_i (Z) \Phi_j(Z)\right] 
= 
\int  \Phi_i (z) \Phi_j(z) \rho(z) dz
= 
\delta_{ij}
\]
Where $\Phi_i$ are Hermite or Legendre polynomials for
Gaussian/Uniform random variables, respectively.
Given this expansion, statistics of the temperature field
are accessible using the orthogonality of the basis functions
\[
m(x) =  E[u(x)] = E[u(x) \Phi_0(Z)] = u_0(x)
\qquad
\Phi_0(Z) = 1
\]
\[
k(x,x') = E[ (u(x)-m(x)) (u(x')-m(x')) ] = \sum_{k=1}^M u_k(x) u_k(x')
\]
Various numerical approaches, $N_i$, are available to compute 
the probability distribution of the QOI. Each \textit{numerical method}
has tradeoffs in terms of computational expense, but in theory
should converge to the exact solution of \eqn{BaseModelOverview}.
\[
\begin{split}
\text{N}_1:  \qquad &
\text{Intrusive Galerkin}
\\
\text{N}_2:  \qquad &
\text{Collocation}
\\
\text{N}_3: \qquad &
\text{MCMC}
\end{split}
\]
%%%%%%%%%%%%%%%%%%%%%%%%%%%%%%%%%%%%%%%%%%%%%%%%%%%%%%%%%%%%%%%%%%%%%%%%%%%
\subsection{Gaussian Process Approximation}
%%%%%%%%%%%%%%%%%%%%%%%%%%%%%%%%%%%%%%%%%%%%%%%%%%%%%%%%%%%%%%%%%%%%%%%%%%%
Alternatively, consider the case where the Karhunen Loeve
expansion is given in  terms of 
\textit{mutually independent} Gaussian random variables
$(Z_1,Z_2,...Z_d)$, $Z_k = \mathcal{N}(m_k,\sigma_k)$.
In this case we will formulate a surrogate model $M_2$
for the base model, $M_1$ \eqn{BaseModelOverview}, presented above.
\textit{Note, that as opposed to the different numerical methods $N_1$,
$N_2$, and $N_3$, the surrogate model $M_2$ is NOT expect to converge to
a solution of the base model $M_1$ and is justified as calling a 'new'
model.}
\[
\begin{split}
\text{M}_1:  \qquad &
\text{Non-Gaussian Statistics}
\\
\text{M}_2: \qquad &
\text{GP}
\end{split}
\]
The model may be written as the usual linear system, $Lu=q_0$, plus 
a non-Gaussian residual, $r(u)$.
\[
  \underbrace{
  -\nabla \cdot ( k_0(\textbf{x}) \nabla u) 
  + \omega_0(\textbf{x}) c_{blood} (u - u_a )
}_{L u}
 = q_0(\textbf{x},t) 
  - 
  \underbrace{
     \sum_{k=1}^d 
    \left(
        -\nabla \cdot ( k_k(\textbf{x})Z_k  \nabla u) 
        + \omega_k(\textbf{x}) Z_K  c_{blood} (u - u_a )
        - q_k(x) Z_k 
    \right)
              }_{r(u)}
\]
As a surrogate model,
classical literature typically assumes that the residual $r(u)$ is too
expensive to compute and models the residual as a projection onto the
Gaussian Basis.  
\[
\epsilon = \sum_{k=1}^d \epsilon_k Z_k
\qquad
\|r(u) - \epsilon \| = \inf_{g \in V} \|r(u) - g\|
\qquad
\Phi_0(Z) = 1,
\Phi_1(Z) = Z_1,
\Phi_2(Z) = Z_2,...,
\Phi_d(Z) = Z_d
\]
\[
V \equiv \left\{ f: E[f^2] < \infty \right\}
\]
Under the Gaussian process assumption, the solution
may be expressed in a similar form
\[
u = u_0 + \sum_{k=1}^d u_k Z_k
\]
By direct substitution and using the orthogonality of the Gaussian basis
a system of uncoupled equations may be derived that involves the usual
deterministic solution plus terms that provide for the Gaussian
fluctuations.
\[
E\left[
\left(
L \left[u_0 + \sum_{k=1}^d u_k Z_k\right]
 =
q_0 + \sum_{k=1}^d \epsilon_k Z_k
\right)
\Phi_k
\right]
\qquad
k = 0,...,d
\]
\[
L u_0 = q 
\qquad
L u_k = \epsilon_k 
\qquad k = 1, ..., d
\]
The mean and covariance of the Gaussian field are thus
\[
m(x) =  E[u(x)] = E[u(x) \Phi_0(Z)] = u_0(x)
\qquad
\Phi_0(Z) = 1
\]
\[
k(x,x') = E[ (u(x)-m(x)) (u(x')-m(x')) ] = \sum_{k=1}^M u_k(x) u_k(x')
\]
Thermal conduction and perfusion appear as \textit{hyper-parameters}
in the covariance function.
Further,
the marginal likelihood of this approach is analytic
and may be used for model selection of dominant covariance terms.
\[
p(D|M_k) = 
 - \underbrace{\frac{1}{2} y^\top K_y^{-1} y }_\text{data fit} -
\underbrace{\frac{1}{2} |K_y|}_\text{complexity}
\]
%%%%%%%%%%%%%%%%%%%%%%%%%%%%%%%%%%%%%%%%%%%%%%%%%%%%%%%%%%%%%%%%%%%%%%%%%%%
\subsubsection{Classical and Low Rank Approximations}
%%%%%%%%%%%%%%%%%%%%%%%%%%%%%%%%%%%%%%%%%%%%%%%%%%%%%%%%%%%%%%%%%%%%%%%%%%%
Classically, the modeling errors are typically 
taken as a Dirac delta functional at a particular point in space (or
finite element node) and the solution is the corresponding fundamental
solution, $G(x,x_k)$
\[
L u_k = \epsilon_k  = \delta(x - x_k)
\quad k = 1, ..., d
\qquad
\Rightarrow
\qquad
u_k = G(x,x_k)
\]
Under these assumptions,
an analytical form for the mean and covariance may be derived
(See Rasmussen 2006, eqn 4.13).
\begin{equation} \label{PhysicsBasedCovarianceFunction}
m(x) = \int_0^t \int_D \Phi(x-y,t-s) Q_\text{laser}(y,s) dy ds
\quad
k(x,x') = \sum_{k=1}^M  G(x,x_k) G(x',x_k)
\end{equation}
\paragraph{Low Rank Approximation} 
Low rank approximations of the covariance lead to efficient FMM methods~\cite{fong2009black}.
As a low rank approximation
the solution may be understood as a single constant perturbation to the temperature
field.
\[
L \left[u_0 + u_1 Z_1\right]
= 
q_0 +  \epsilon Z_1
\qquad 
\epsilon \in \mathbb{R}
\]
Two linear system solves are needed
\[
L u_0 = q 
\qquad
L u_1 = \epsilon 
\]
\textbf{Notice in spherical coordinates that an analytic form for the low
rank covariance may also be derived. IE solve pde for constant right hand
side and get analytic expression for $u_1$.}
\[
k(x,x') = u_1(x) u_1(x')
\]
The covariance kernel will be in terms of $\rho$, $k$, and $\omega$, NOT
$\mu_\text{eff}$
\paragraph{Residual Low Rank Approximation} 
As a low rank approximation
the solution may be understood as a potential nonlinear perturbation, 
due to Arrhenius damage, $\Omega$, to the temperature
field resulting from each term in the residual.
\[
 k(x)      =      k_0(x) + \delta    k(x,\Omega)   Z_1
     \qquad
 \omega(x) = \omega_0(x) + \delta \omega(x,\Omega) Z_2
     \qquad
 q(x)      =      q_0(x) + \delta    q(x,\Omega)   Z_3
\]
\[
L \left[u_0 + u_1 Z_1 + u_2 Z_2 + u_3 Z_3\right]
= 
q_0 
 \underbrace{
    + \underbrace{\nabla \cdot ( \delta k(x,\Omega)  \nabla u_0) }_{ \epsilon_1 }Z_1
    - \underbrace{\delta \omega(x,\Omega) c_{blood} (u_0 - u_a ) }_{ \epsilon_2 }Z_2
    + \underbrace{                \delta   q(x,\Omega)           }_{ \epsilon_3 }Z_3
            }_{r(u_0)}
\]
The approximate residual is bounded by
\[
   \|r(u_0) - r(u)\|  \quad < \quad  ???
\]
Four linear system solves are needed
\[
L u_0 = q 
\qquad
L u_1 = \nabla \cdot ( \delta k(x,\Omega)  \nabla u_0) 
\qquad
L u_2 =-\delta \omega(x,\Omega) c_{blood} (u_0 - u_a ) 
\qquad
L u_3 =                   \delta q(x,\Omega)                    
\]
Notice that the covariance includes the contributions from \textbf{MEAN AND UNCERTAINTIES}
in the thermal conduction, perfusion, and optical properties directly as
hyperparameters
\[
k(x,x') = u_1(x) u_1(x') + u_2(x) u_2(x') + u_3(x) u_3(x') 
\]
%%%%%%%%%%%%%%%%%%%%%%%%%%%%%%%%%%%%%%%%%%%%%%%%%%%%%%%%%%%%%%%%%%%%%%%%%%%
\subsubsection{FMM and Fast Gauss Transform}
%%%%%%%%%%%%%%%%%%%%%%%%%%%%%%%%%%%%%%%%%%%%%%%%%%%%%%%%%%%%%%%%%%%%%%%%%%%
Consider using FMM \texttt{http://www.bu.edu/exafmm/} for each solve.
Each prediction estimate involves solution to a linear system of equations
of the form
\[
   K x= y
\]
The variance for each point requires the linear system solve with multiple
RHS (=N)
\[
  K X = B
\]
The underlying correlation function is the form of a summation of
Fundamental solutions~\eqn{PhysicsBasedCovarianceFunction}.
\[
  k(x,x') = \sum_{k=1}^M  G(x,x_k) G(x',x_k)
\]
or low rank approximation
\[
  k(x,x') = u_1(x) u_1(x')
\]
The matrix vector multiply is of a form suitable for multipole expansion 
\[
f(x) =  \sum_{i=1}^N c_i k(x,x_i) 
 \underbrace{\; = \;}_\text{FMM}
     \sum_{k=1}^M  G(x,x_k) \sum_{i=1}^N c_i  G(x_i,x_k)
\]
or low rank expansion~\cite{fong2009black}, ie $M << N$.
\[
f(x) =  \sum_{i=1}^N c_i k(x,x_i) 
 \underbrace{\; = \;}_\text{FMM}
     \sum_{k=1}^M  u_k(x) \sum_{i=1}^N c_i  u_k(x_i)
\]
As opposed to the Chebyshev approximation of~\cite{fong2009black},
the matrix vector product of the FMM expansion is approximation with the
FEM, SEM, or BEM basis used for $u_1$.

For Gaussian kernels a Fast Gauss transform exists \cite{yang2005efficient}.


Implementations may be found at:
\texttt{http://www.umiacs.umd.edu/labs/cvl/pirl/vikas/Software/IFGT/IFGT\_code.htm}

A good preconditioner is needed for the $\mathcal{O}(N)$ linear
system solves needed, Section~\ref{MultipleRHS}.


%%%%%%%%%%%%%%%%%%%%%%%%%%%%%%%%%%%%%%%%%%%%%%%%%%%%%%%%%%%%%%%%%%%%%%
\begin{thebibliography}{}
\bibitem[Carp 2004]{carp2004rtd}
Carp, S.A. and Prahl, S.A. and Venugopalan, V.
\newblock {Radiative transport in the delta-[bold P] approximation: accuracy of fluence rate and optical penetration depth predictions in turbid semi-infinite media}. 
\newblock Journal of Biomedical Optics

\bibitem[Modest 2003]{Modest2003}
M. Modest 
\newblock {\em Radiative Heat Transfer}.
\newblock New York: Academic Press, 2003.

\bibitem[Welch 1995]{Welch95}
A.~J. Welch and M.~J.~C. van Gemert.
\newblock {\em Optical-Thermal Response of Laser-Irradiated Tissue}.
\newblock New York: Plenum Press, 1995.

\bibitem[Fasano 2010]{Fasano2010}
Fasano, A. and H{\"o}mberg, D. and Naumov, D.
\newblock {On a mathematical model for laser-induced thermotherapy}.
\newblock Applied Mathematical Modelling 2010

\bibitem[Fong and Darve, 2009]{fong2009black}
Fong, W. and Darve, E. (2009).
\newblock The black-box fast multipole method.
\newblock {\em Journal of Computational Physics}, 228(23):8712--8725.

\bibitem[Yang et~al., 2005]{yang2005efficient}
Yang, C., Duraiswami, R., and Davis, L.~S. (2005).
\newblock Efficient kernel machines using the improved fast gauss transform.
\newblock In {\em Advances in neural information processing systems}, pages
  1561--1568.

\bibitem{boyce1992elementary}
William~E Boyce, Richard~C DiPrima, and Charles~W Haines.
\newblock {\em Elementary differential equations and boundary value problems},
  volume~9.
\newblock Wiley New York, 1992.

\bibitem{Giordano2010}
Mauricio~a Giordano, Gustavo Gutierrez, and Carlos Rinaldi.
\newblock {Fundamental solutions to the bioheat equation and their application
  to magnetic fluid hyperthermia.}
\newblock {\em International journal of hyperthermia : the official journal of
  European Society for Hyperthermic Oncology, North American Hyperthermia
  Group}, 26(5):475--84, January 2010.

\bibitem{Vyas1992}
Reeta Vyas.
\newblock {Green’s function solution to the tissue bioheat equation}.
\newblock {\em Medical Physics}, 19(5):1319, 1992.

\bibitem{Deng2002}
Zhong-Shan Deng and Jing Liu.
\newblock {Analytical Study on Bioheat Transfer Problems with Spatial or
  Transient Heating on Skin Surface or Inside Biological Bodies}.
\newblock {\em Journal of Biomechanical Engineering}, 124(6):638, 2002.

\end{thebibliography}

\end{document}
